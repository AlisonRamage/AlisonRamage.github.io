\documentclass[12pt]{article}
\pagestyle{empty}
\usepackage[dvips]{graphics}
\usepackage{amssymb}
\usepackage{lscape}
\setlength{\textheight}{24.0cm}
\setlength{\textwidth}{16.5cm}
\setlength{\parindent}{0.0cm}
\setlength{\topmargin}{-2.0cm}
\setlength{\oddsidemargin}{-0.5cm}

\begin{document}
\begin{center}
{\large An Introduction to \LaTeX : practical exercises}
\end{center}
\begin{enumerate}
\item Go to the webpage
\begin{center}
\texttt{https://alisonramage.github.io/latex\_course/}
\end{center}
and copy the files \texttt{LaTeXnotes.tex, fig1.png}  and \texttt{wpdoc.tex}
 to your own filespace.

\item Start \texttt{TeXworks} from the Windows menu. Open the 
files LaTeXnotes.tex and wpdoc.tex using \texttt{TeXworks} and process them
to create PDF files of the course notes and sample document. 

\item Create a new \LaTeX\, document as shown on the slide, and view the 
resulting document on the screen. (When you LaTeX a new document for the first 
time in \texttt{TeXworks}, you will be asked to save it using a new name.)

\item Edit your \LaTeX\, document from part 3 so that it reproduces the text 
below. (Note: the LaTeX command to produce the typset \LaTeX\, name is 
\verb+\LaTeX+.)\\
\rule{16cm}{0.01cm}
\medskip

This is my first document in \LaTeX! I am very \textit{excited} 
about it, so I want to 
\begin{center}
\textbf{shout out loud}.
\end{center}

\bigskip
\item Add the piece of mathematics below to your document.\\
\rule{16cm}{0.01cm}

Find $x$ and $y$ such that
\begin{displaymath}
\sin{x}+\cos{y}=\frac{3(\pi+1)}{25^2}
\end{displaymath}
with $x\ne 0$ and $y>5$.

\bigskip
\item Add a title and author to your document (you can use the commands in 
wpdoc.tex as a template).

\bigskip
\item Edit your document so that the equation above has a number, then add the 
reference sentence and citation.\\
\rule{16cm}{0.01cm}


Find $x$ and $y$ such that
\begin{equation}
\sin{x}+\cos{y}=\frac{3(\pi+1)}{25^2}
\label{eq1}
\end{equation}
with $x\ne 0$ and $y>5$. Equation (\ref{eq1}) can be used to show that $x>0$ 
\cite{faketext}.
\begin{thebibliography}{99}
\bibitem{faketext}{\textsc{A. Author}, \textit{Mathematics Text}, 
\textbf{Strathclyde Publishing}, 2016.}
\end{thebibliography}

\end{enumerate}
\newpage
\begin{center}
\fbox{\large An Introduction to \LaTeX: Course Assignment}
\end{center}
Imagine that you are applying for follow-up funding for your PhD studies. 
Use \LaTeX\, to prepare a two-page document to support your application. 
This should comprise
\begin{itemize}
\item a one-page CV, listing  your qualifications and achievements to date;
\item a brief description of your PhD project, with a summary of progress and a 
timeline of the key milestones to come.
\end{itemize}
Try to make your document more attractive to read by including lists and
tabbing (to line up the material in columns) where appropriate, and varying the 
fonts and textsize.
\end{document}
