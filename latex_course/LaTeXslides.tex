\documentclass{beamer}
%\documentclass[handout]{beamer}
\usepackage[latin1]{inputenc}
\usetheme{Warsaw}

\usepackage{color}
\usepackage{theorem}
\usepackage{fancybox}
\usepackage{listings}
\newcommand{\bc}{\begin{center}}
\newcommand{\ec}{\end{center}}
\newcommand{\bd}{\begin{displaymath}}
\newcommand{\ed}{\end{displaymath}}
\newcommand{\bi}{\begin{itemize}}
\newcommand{\ei}{\end{itemize}}
\newcommand{\lx}{{\LaTeX} }
\newcommand{\bs}{$\backslash$}
\def\mynormal{\vspace*{-0.5cm}}

\title[An Introduction to \lx]
{An Introduction to \lx
}
\author{Dr Alison Ramage, \texttt{A.Ramage@strath.ac.uk}}
\institute{Dept of Mathematics and Statistics, LT10.11 Livingstone Tower}
\date{}
\begin{document}
%%%%%%%%%%%%%%%
\begin{frame}
\titlepage
\end{frame}
%%%%%%%%%%%%%%%%%%%%%%%%%%%%%%%%%%%%%%%%%%%%%%%%%%%%%%%%%%%%%%%%%%%%%%%%%%%%%%
\begin{frame}{References}
\begin{itemize}
\item
\textcolor{blue}{
{\em Learning \lx}}\\
Griffiths and  Higham, SIAM 2016\\
(second edition)\\
\medskip
\item
\textcolor{blue}{
{\em \lx: A Document Preparation System}}\\
Lamport, Addison-Wesley 1994 \\
\medskip

\item 
\textcolor{blue}{
{\em The \lx Companion}}\\
Mittelbach and Goossens, Addison-Wesley 2004\\

\medskip
\item A wealth of internet sources!
\end{itemize}
\end{frame}
%%%%%%%%%%%%%%%%%%%%%%%%%%%%%%%%%%%%%%%%%%%%%%%%%%%%%%%%%%%%%%%%%%%%%%%%%%%%%%
\begin{frame}{Basics}
\mynormal
\medskip
\begin{itemize}
\item \textcolor{blue}{typesetting} package (not \textcolor{red}{WYSIWYG})
\medskip
\item based on the \textcolor{blue}{\TeX}\, program by Don Knuth (Stanford, 
1978)\\
\medskip
\item \textcolor{blue}{\lx} written by Leslie Lamport specifically for maths
\end{itemize}
\begin{quotation}
\textcolor{red}{
``a house built with wood and nails of \TeX''
}
\end{quotation}
\begin{itemize}
\pause
\item cross-references and citations done \textcolor{blue}{automatically}
\medskip
\item convenient for electronic publication
\medskip
\item \textcolor{blue}{public domain} software (i.e.\ \textcolor{red}{free}!)
\end{itemize}
\pause
\bc
\textcolor{red}{
``Although you can format an equation almost any way you want with \lx, you have
to work harder to do it wrong.''}
\ec
\end{frame}
%%%%%%%%%%%%%%%%%%%%%%%%%%%%%%%%%%%%%%%%%%%%%%%%%%%%%%%%%%%%%%%%%%%%%%%%%%%%%%
\begin{frame}{Running \lx}
%\mynormal
\bi
\item First \textcolor{red}{create a file} containing \lx commands, then 
\textcolor{red}{compile} it to process the commands.
\pause
\medskip
\item Can be run from the command line or a \textcolor{blue}{Command Prompt} 
window.
\medskip
\pause
\item Usual to use a customised \lx User `front end'.
Examples:
\bi
\item \textcolor{blue}{\texttt{TeXworks}} (for Windows operating systems) with
\textcolor{blue}{MiK\TeX};
\item \textcolor{blue}{\texttt{kile}} (for Mac OS X and Linux operating 
systems);
\item \textcolor{blue}{\texttt{overleaf}} (online at 
\texttt{https://www.overleaf.com/}).
\ei
\medskip
\pause
\item All have many options which can be customised to your preferences 
once you have some experience.
\medskip
\pause
\item In this course we will use \textcolor{blue}{\texttt{TeXworks}}.
\medskip
\pause
\item May need to configure PS or PDF viewing options for local 
architecture.
\ei
\end{frame}
%%%%%%%%%%%%%%%%%%%%%%%%%%%%%%%%%%%%%%%%%%%%%%%%%%%%%%%%%%%%%%%%%%%%%%%%%%%%%%
\begin{frame}{Overview}
\mynormal
\medskip
\bc
create a \lx .tex file\\
\textcolor{blue}{doc.tex}
\ec
\vspace{-0.5cm}
\begin{center}
\scalebox{0.1}{\includegraphics{downarrow.eps}}
\end{center}
\vspace{-0.8cm}
\bc
process the .tex file\\
\textcolor{blue}{LaTeX button}
\ec
\vspace{-2.0cm}
\begin{flushright}
\scalebox{0.1}{\includegraphics{uparrow.eps}}
\parbox{3cm}{\textcolor{red}{correct any\\ \lx  errors}}
\end{flushright}
\vspace{0.1cm}
\begin{center}
\scalebox{0.1}{\includegraphics{downarrow.eps}}
\end{center}
\vspace{-0.7cm}
\bc
preview the .dvi file\\
\textcolor{blue}{ViewDVI button}
\ec
\vspace{-3.8cm}
\begin{flushleft}
\hspace*{1.2cm}\scalebox{0.1}{\includegraphics{uparrow.eps}}
\end{flushleft}
\begin{flushleft}
\parbox{3cm}{\bc\textcolor{red}{correct any\\ style, layout,\\content 
errors}\ec}
\end{flushleft}
\begin{flushleft}
\hspace*{1.2cm}\scalebox{0.1}{\includegraphics{uparrow.eps}}
\end{flushleft}
\vspace{-1cm}
\begin{center}
\scalebox{0.1}{\includegraphics{downarrow.eps}}
\end{center}
\vspace{-0.7cm}
\bc
convert the .dvi file for printing\\
\textcolor{blue}{DVItoPS or DVItoPDF button}
\ec
\begin{center}
\scalebox{0.1}{\includegraphics{downarrow.eps}}
\end{center}
\vspace{-0.7cm}
\bc
view and print the .ps or .pdf file\\
\textcolor{blue}{ViewDVI or ViewPDF button}
\ec
\end{frame}
%%%%%%%%%%%%%%%%%%%%%%%%%%%%%%%%%%%%%%%%%%%%%%%%%%%%%%%%%%%%%%%%%%%%%%%%%%%%%%
\begin{frame}{Notes}
\mynormal
\medskip
\begin{itemize}
%\item A \lx file must have the extension \textcolor{blue}{.tex}
\item If you are using internal references, you may get the message
\begin{flushleft}
\texttt{LaTeX Warning: Label(s) may have changed.
Rerun to get cross-references right.}
\end{flushleft}
You must then \textcolor{red}{recompile} the file by clicking the 
\textcolor{blue}{LaTeX} button again.
\pause
\medskip
\item Several different ways of producing a PDF file:
\bi
\item use the \textcolor{blue}{LaTeX} button then 
\textcolor{blue}{DVItoPDF};
\item use the \textcolor{blue}{PDFLaTeX} button to compile the file and 
produce a PDF file directly;
\item use the \textcolor{blue}{LaTeX} button then \textcolor{blue}{DVItoPS} 
followed by \textcolor{blue}{PStoPDF}.
\ei
\pause
\medskip
\item The \textcolor{blue}{previewing} stage is very important: printing is
\textcolor{red}{expensive}!

\end{itemize}
\end{frame}
%%%%%%%%%%%%%%%%%%%%%%%%%%%%%%%%%%%%%%%%%%%%%%%%%%%%%%%%%%%%%%%%%%%%%%%%%%%%%%
\begin{frame}{TeXworks Overview}
\mynormal
\medskip
\begin{enumerate}
\item Load or create a file containing \lx commands. 
\begin{itemize}
\item The file must have suffix \texttt{.tex}.
\end{itemize}
\bigskip
\pause
\item Process the .tex file to convert it to a PDF file.
\begin{itemize}
\item Choose \textcolor{red}{pdfLaTeX} from the left-hand drop-down menu, then 
press the \textcolor{red}{green arrow} next to it.
\end{itemize}
\bigskip
\pause
%\item For the first time with a new file, you will have to supply a \textcolor{red}{filename} and save location.
%\bigskip
%\pause
\item If there are no errors, a PDF file will appear containing your document.
\bigskip
\pause
\item If there is an error in your file, press the \textcolor{red}{red cross} button to cancel the process, correct the file 
and try again.
\end{enumerate}
\begin{flushright}
\framebox{Exercises 1 and 2}
\end{flushright}
\end{frame}
%%%%%%%%%%%%%%%%%%%%%%%%%%%%%%%%%%%%%%%%%%%%%%%%%%%%%%%%%%%%%%%%%%%%%%%%%%%%%%

\begin{frame}{In your .tex file}
\mynormal
\medskip
\begin{itemize}
\pause
\item\textcolor{red}{ALL} \lx commands begin with a \,\,
$\textcolor{blue}{\backslash}$\,\,
and are case-sensitive.
\pause
\item Special characters:
\bc
\textcolor{blue}{
\% $\ast$ \$ \& \_ \{ \} \~{} \^{} $\backslash$}
\ec
will be interpreted as \lx control characters.
\pause
\item \textcolor{blue}{\%} acts as a `comment' symbol: anything on the line
after a \textcolor{blue}{\%} sign is ignored.
\pause
\item Some commands have \textcolor{blue}{arguments}:

\begin{center}
enclosed in \textcolor{red}{$\{\,\,\}$}: \textcolor{blue}{compulsory}\\
\end{center}
\begin{center}
enclosed in \textcolor{red}{$[\,\,]$}: \textcolor{blue}{optional}\\
\end{center}
\end{itemize}
\end{frame}

\begin{frame}{Document structure}
\mynormal
\medskip
Each document has two parts:

\begin{itemize}
\item \textcolor{red}{PREAMBLE}\\
This sets up the document class, type size, 
page settings etc. The first line must be

\begin{center}
\textcolor{blue}{\texttt{$\backslash$documentclass\{STYLE\}}}
\end{center}

where STYLE is one of
\begin{itemize}
\item \textcolor{blue}{article} (includes sections and subsections)
\item                \textcolor{blue}{report}  (includes chapters)
    \item            \textcolor{blue}{book}    (includes volumes)
        \item        \textcolor{blue}{letter}
            \item    \textcolor{blue}{beamer} (for slides)
            \item    \textcolor{blue}{a0poster} (for posters)
\item \textcolor{blue}{\ldots}
\end{itemize}
\end{itemize}
\end{frame}
\begin{frame}{Preamble (cont.)}

\mynormal
\medskip
The {preamble} may also contain 
\begin{itemize}
\item commands which define page size, margins etc,
e.g.

{\tt
$\backslash$setlength\{$\backslash$textheight\}\{18.0cm\}\\
$\backslash$setlength\{$\backslash$topmargin\}\{1.2in\}\\
$\backslash$pagestyle\{empty\}\\ }
\pause
\item inclusion of any \textcolor{blue}{packages}, e.g.

{\tt 
$\backslash$usepackage\{amssymb\}\\ }
\pause
\item user-defined \textcolor{blue}{new commands}, e.g.

{\tt 
$\backslash$newcommand\{$\backslash$fe\}\{finite element method\}\\ }
\pause
\item user-defined \textcolor{blue}{changes} to default style, e.g.

{\tt 
$\backslash$renewcommand\{$\backslash$baselinestretch\}\{1.5\}\\ }
\end{itemize}
\end{frame}

\begin{frame}{Document structure (cont.)}
\mynormal
\medskip
\begin{itemize}
\item \textcolor{red}{DOCUMENT BODY}\\
This contains the \textcolor{blue}{\lx} commands to produce the document 
text.\\
\medskip
{\tt
$\backslash$begin\{document\}             \\
THE DOCUMENT TEXT GOES HERE.\\
$\backslash$end\{document\}\\
}
\medskip
\pause
\begin{itemize}
\item 
A document can be \textcolor{blue}{subdivided} using\\
\medskip

{\tt
$\backslash$chapter\{...\}\\
$\backslash$section\{...\}\\
$\backslash$subsection\{...\}\\
$\backslash$subsubsection\{...\}\\
$\backslash$Appendix
}
\end{itemize}
\end{itemize}
\end{frame}

\begin{frame}{Document structure (cont.)}
\mynormal
\medskip
\begin{itemize}
\item 
Separate \LaTeX \,\,input files can be \textcolor{blue}{included} using\\
\medskip
{\tt
$\backslash$input\{\textit{filename}\}}\\
\medskip
\item 
A \textcolor{blue}{table of contents} can be included using\\
\medskip
{\tt
$\backslash$tableofcontents}\\
\bigskip
\bigskip
Sample document body:\\
\bigskip
{\tt
$\backslash$tableofcontents\\
$\backslash$newpage\\
 $\backslash$input\{chapter1\}\\
$\backslash$newpage\\
 $\backslash$input\{chapter2\}\\
}
\end{itemize}
\end{frame}


\begin{frame}{SUMMARY}
\mynormal
\medskip
\begin{itemize}
\item A valid \lx  document:\\
\medskip
{\tt
$\backslash$documentclass\{article\}\\
$\backslash$begin\{document\}\\
This is my first document.\\
$\backslash$end\{document\}\\}
\end{itemize}
\bigskip
\pause
\bc
\textcolor{blue}{This is all you need, but\ldots}
\ec
\pause
\bc
\textcolor{blue}{\ldots hopefully your documents will be a little more
sophisticated!}
\ec
\begin{flushright}
\framebox{Exercise 3}
\end{flushright}
\end{frame}
\begin{frame}{Document style}
\mynormal
\begin{itemize}
\item A \textcolor{blue}{title} can be created with
{\tt $\backslash$maketitle} which uses
\bc
{\tt\bs title\{...\}\qquad
\bs author\{...\}\qquad
\bs date\{\bs today\} }
\ec
\medskip
\pause
\item Varying text \textcolor{blue}{font}:
\bc
\textit{\bs textit\{...\}}\qquad
\textsc{\bs textsc\{...\}}\qquad 
\textsf{\bs textsf\{...\}}\\
\textbf{\bs textbf\{...\}}\qquad 
\texttt{\bs texttt\{...\}}
\ec
\medskip
\pause
\item Varying text \textcolor{blue}{size}:
\bc{\tt
\tiny \bs tiny\qquad \scriptsize\bs scriptsize\qquad \footnotesize\bs 
footnotesize\qquad 
\normalsize \bs normalsize \qquad
\large \bs large\qquad 
\Large \bs Large\qquad \LARGE \bs LARGE\qquad \huge \bs huge\qquad \Huge\bs 
Huge}
\ec
\end{itemize}
\end{frame}

\begin{frame}[fragile]{Some useful \lx concepts (1)}
\mynormal
\medskip
\begin{itemize}
\item \textcolor{blue}{environments}
\end{itemize}
\begin{lstlisting}
\begin{center}...\end{center}
\begin{itemize}...\end{itemize}
\begin{enumerate}...\end{enumerate}
\begin{description}...\end{description}
\begin{tabbing}...\end{tabbing}
\begin{tabular}...\end{tabular}
\begin{table}...\end{table}
\begin{figure}...\end{figure}
\begin{quote}...\end{quote}
\begin{verse}...\end{verse}
\end{lstlisting}
\begin{flushright}
\framebox{Exercise 4}
\end{flushright}
\end{frame}

\begin{frame}{Some useful \lx concepts (2)}
\mynormal
\medskip
\begin{itemize}
\item \textcolor{blue}{math mode}\\

All mathematics must be typeset in \textcolor{blue}{math mode}. This can be 
done 
using \textcolor{red}{dollar signs} \textcolor{blue}{\$...\$}, e.g.
\bc
{\tt Let \$x\$ be a real number.}\\
\medskip
 Let $x$ be a real number.
\ec
\medskip
\pause
\item \textcolor{blue}{environments in math mode}\\
\bc
\tt
$\backslash$begin\{displaymath\}...$\backslash$end\{displaymath\}\\
$\backslash$begin\{equation\}...$\backslash$end\{equation\}\\
$\backslash$begin\{array\}...$\backslash$end\{array\}\\
$\backslash$begin\{eqnarray\}...$\backslash$end\{eqnarray\}\\
$\backslash$begin\{eqnarray$\ast$\}...$\backslash$end\{eqnarray$\ast$\}\\
\ec
\end{itemize}
\end{frame}


\begin{frame}{Examples of maths commands (1)}
\mynormal
\begin{itemize}
\item \textcolor{blue}{maths fonts}:
\bc
$\mathcal{A}$\qquad
$\mathrm{A}$\qquad
$\mathit{A}$\qquad
$\mathsf{A}$\qquad
$\mathbf{A}$\qquad
$\mathtt{A}$\\
{\tt
{\bs mathcal\{A\}},
{\bs mathrm\{A\}},
{\bs mathit\{A\}},\\
{\bs mathsf\{A\}},
{\bs mathbf\{A\}},
{\bs mathtt\{A\}}}
\ec
\pause
\medskip
\item \textcolor{blue}{Greek letters}:\qquad
$\alpha,\quad\beta, \quad \gamma, \quad\Gamma, \quad\sigma, \quad\Sigma$
\bc
{\tt
\bs alpha,\bs beta,\bs gamma,\bs Gamma,\bs sigma,\bs Sigma
}
\ec
\pause
\medskip
\item \textcolor{blue}{symbols}:\qquad
$\ne,\quad\Leftrightarrow,\quad\in ,\quad\sim,\quad\nabla,\quad\partial$
\bc
{\tt
\bs ne,\bs Leftrightarrow,\bs in,\bs sim,\bs nabla,\bs partial
}
\ec
\pause
\medskip
\item \textcolor{blue}{variable-sized symbols}:\qquad
$\int,\quad\oint,\quad\sum,\quad\left[,\quad\right]$
\bc
{\tt 
\bs int,\quad\bs oint,\quad\bs sum,\quad\bs left[, \quad\bs right]}
\ec
\end{itemize}
\end{frame}

\begin{frame}{Examples of maths commands (2)}
\mynormal
\medskip
\begin{itemize}
\item \textcolor{blue}{subscripts} and \textcolor{blue}{superscripts}:\qquad
\bd
x_1,\quad y_{ij},\quad z^{n+1},\quad\lim_{x\rightarrow -1},\quad \int_1^\infty
\ed
\bc
{\tt x\_1,\quad y\_\{ij\},\quad z\^{}\{n+1\},\\
\quad\bs lim\_\{x\bs rightarrow -1\},\quad \bs int\_1\^{}\bs infty }
\ec
\bigskip
\pause
\item \textcolor{blue}{fractions}:
\bd
x=\frac{3+\sin{t}}{t^2},\qquad y=\frac{\partial x}{\partial t}
\ed
\bc
{\tt
x=\bs frac\{3+\bs sin\{t\}\}\{t\^{}2\}\\ y=\bs frac\{\bs partial x\}\{\bs 
partial t\}}
\ec
\end{itemize}
\end{frame}
\begin{frame}{Arranging formulae}
\mynormal
\medskip
\begin{itemize}
\item \textcolor{blue}{arrays}:\qquad
$A=\left[\begin{array}{ccc}
1 & 1 & 1\\
x & y & z\\
x^2 & y^2 & z^2
\end{array}\right]$\\
\bigskip
{\tt
A=\bs left[\bs begin\{array\}\{ccc\}\\
1 \& 1 \& 1\bs \bs  x \& y \& z\bs \bs x\^{}2 \& y\^{}2 \& z\^{}2\\
 \bs end\{array\} \bs right]
}
\pause
\medskip
\item \textcolor{blue}{equation arrays}:
\vspace{-0.5cm}
\begin{eqnarray}
x&=&17+p^2-3p^5\\
y&=&\alpha -\theta\nonumber
\end{eqnarray}
{\tt \bs begin\{eqnarray\}\\
x\&=\&17+p\^{}2-3p\^{}5\bs \bs \\
y\&=\&\bs alpha - \bs theta\bs nonumber\\
\bs end\{eqnarray\}}
\end{itemize}
\end{frame}
\begin{frame}[fragile]{A simple table}
\mynormal
\bigskip
\begin{center}
\begin{tabular}{||l||c|c|c|c|c|c|}
\hline
Team & Played & W & D & L & Goals & Points\\\hline\hline
Aberdeen & 2 & 2 & 0 & 0 & +10 & 6\\\hline
Celtic & 2 & 0 & 1 & 1& -5 & 1 \\\hline
Rangers & 2 & 0 & 1 & 1& -5 & 1 \\\hline
\end{tabular}
\end{center}
\begin{lstlisting}
\begin{center}
\begin{tabular}{||l||c|c|c|c|c|c|}
\hline
Team & Played & W & D & L & Goals & Points\\
\hline\hline
Aberdeen & 2 & 2 & 0 & 0 & +10 & 6\\\hline
Celtic & 2 & 0 & 1 & 1& -5 & 1 \\\hline
Rangers & 2 & 0 & 1 & 1& -5 & 1 \\\hline
\end{tabular}
\end{center}
\end{lstlisting}
\begin{flushright}
\framebox{Exercise 5}
\end{flushright}
\end{frame}

\begin{frame}{Cross-referencing}
\mynormal
\medskip
\lx can \textcolor{red}{automatically} number equations, references etc
\begin{itemize}
\medskip
\item to label an equation: \qquad\textcolor{blue}{\tt \bs label\{...\}}\\
\begin{equation}
x=1
\label{eq1}
\end{equation}
{\tt
\bs begin\{equation\}\\
\textcolor{blue}{\bs label\{eq1\}}\\
x=1\\
\bs end\{equation\}}
\pause
\medskip
\item to refer to a label:\qquad\textcolor{blue}{\tt \bs ref\{...\}}\\
\medskip
\bc
Using equation (\ref{eq1}), we see that\ldots
\ec
{\tt
Using equation (\textcolor{blue}{\bs ref\{eq1\}}), we see that\ldots
}
\end{itemize}
\end{frame}
\begin{frame}{Citing reference texts}
\mynormal
\medskip
\begin{itemize}
\item  bibliography:\\
\medskip
{\tt
\textcolor{blue}{
\bs begin\{thebibliography\}\{99\}}\\
BIBLIOGRAPHY ITEMS\\
\textcolor{blue}{
\bs end\{thebibliography\}}
}
\pause
\medskip
\item sample bibliography entry:\\
\medskip{\tt
\textcolor{blue}{
\bs bibitem\{Ramage04\} \{\\
\textcolor{red}{A. Ramage, Famous Book, OUP, 2004.}\\
\textcolor{blue}{\}}\\
} }
\medskip
\pause
\item to refer to a reference text:\qquad\textcolor{blue}{\tt \bs 
cite\{...\}}\\
\medskip
{\tt In \textcolor{blue}{\bs cite\{Ramage04\}} we see that \ldots} 
\pause
\medskip
\item fancier methods available, e.g. \textcolor{red}{bibtex}
\end{itemize}
\end{frame}
\begin{frame}{Including pictures}
\mynormal
\begin{itemize}
\item plots can be in a number of formats, e.g., \textcolor{blue}{PostScript} (.ps),
\textcolor{blue}{Encapsulated PostScript} (.eps), \textcolor{blue}{Portable Network Graphics} (.png), etc.
\pause
\item many different packages available, e.g. 
\textcolor{blue}{psfig}, \textcolor{blue}{epsf} etc
\pause
\item \textsc{Example}
\begin{itemize}
\item in preamble:\\
\texttt{\bs usepackage\{graphicx\}}
\item in text:\\
{\tt \bs begin\{figure\}[ht]\\
\bs begin\{center\}\\
\bs scalebox\{0.3\}\{\bs includegraphics\{fig.png\}\}\\ 
\bs end\{center\}\\
\bs caption\{An example of including a picture.   \bs label\{fig1\}\}\\
\bs end\{figure\}}
\end{itemize}
\item use \textcolor{blue}{scalebox} to change the size of the picture
\end{itemize}
\end{frame}
\begin{frame}{Packages and style files}
\mynormal
\medskip
\begin{itemize}
\item \textcolor{blue}{.sty}, \textcolor{blue}{.cls} files available from many
sources:
\bigskip
\begin{itemize}
\item colleagues and fellow students
\medskip
\item publishers, e.g. 
\textcolor{blue}{siamltex},
\textcolor{blue}{elsart}
\medskip
\item American Mathematical Society, e.g.
\textcolor{blue}{amsfonts},
\textcolor{blue}{amsmath},
\textcolor{blue}{amssymb}
\bc
\textcolor{red}{
$\mathbb{R}$, $\mathbb{Z}$, $\mathbb{C}$\\
{\tt \$\bs mathbb\{R\}\$,
\$\bs mathbb\{Z\}\$,
\$\bs mathbb\{C\}\$}}
\ec
\medskip
\item UK\TeX\,\,  archive\qquad {\tt http://www.tex.ac.uk}
\medskip
\item Google search!
\end{itemize}
\medskip
\item include packages with the 
\textcolor{blue}{\texttt{\bs usepackage\{packagename\}}}
command
\end{itemize}
\end{frame}
\begin{frame}{Slides and presentations}
\mynormal
\medskip
\begin{itemize}
\item \textcolor{blue}{beamer} document class
\bc
{\tt \bs documentclass\{beamer\} }
\ec
\bc
{\tt \bs begin\{frame\}...\bs end\{frame\}}
\ec
\pause
\medskip
\item slide style, colour etc.\ 
can be specified using the \textcolor{blue}{\texttt{\bs theme}} and 
\textcolor{blue}{\texttt{\bs colortheme}} commands. 
\medskip
\item use standard themes or create your own
\medskip
\item \textcolor{blue}{a0poster} document class for posters
\medskip
\item more information about these packages online
\end{itemize}
\end{frame}
\begin{frame}{Support Material}

Available from\\

\bigskip

\begin{tabbing}
xxxxxxxxxxxxxxxxxxxxxxx\= \kill
\texttt{https://alisonramage.github.io/latex\_course/}
\end{tabbing}

\bigskip
\bigskip
\bigskip

\begin{tabbing}
xxxxxxxxxxxxxxxxxxxxxxxxxxxx\= \kill
\textcolor{blue}{$\bullet$} \lx notes \>  
\texttt{LaTeXnotes.tex}\\
\textcolor{blue}{$\bullet$} sample file \>  
\texttt{wpdoc.tex}\\
\textcolor{blue}{$\bullet$} sample figure \>  
\texttt{fig1.png}\\
\textcolor{blue}{$\bullet$} slides from this talk \>  
\texttt{LaTeXslides.tex}\\
\end{tabbing}
\begin{flushright}
\framebox{Exercises 6 and 7}
\end{flushright}

\end{frame}
\end{document}
