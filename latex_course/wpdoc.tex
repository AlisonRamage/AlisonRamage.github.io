% A comment line in your .tex file must start with a percent sign
% PREAMBLE
% this is the minimum necessary in the preamble
\documentclass[12pt]{article}

% PAGE SPECIFICATIONS
% these are optional (otherwise default values will be used)
\setlength{\textheight}{24.0cm}
\setlength{\topmargin}{-1.0cm}

% SETTING NEW ABBREVIATIONS
% to change the definition of an existing command
\renewcommand{\theequation}{\thesection.\arabic{equation}}
% to add your own new command
\newcommand{\fe}{finite element }

% TITLE INFORMATION
\title{On Preconditioning for Finite Element Equations
 on Irregular Grids}
\author{Alison Ramage 
        \thanks{Department of Mathematics, University of Strathclyde,
        Glasgow G1 1XH, United Kingdom.}}
% this will print today's date
\date{\today}

% DOCUMENT BODY
\begin{document}

% PRINTING TITLE 
\maketitle

% ABSTRACT
\begin{abstract}
Preconditioning methods are widely used in conjunction with the
conjugate gradient method for solving large sparse symmetric linear systems
arising from the discretisation of self-adjoint linear elliptic
partial differential equations.
\end{abstract}

% SECTION 1
\section{Introduction}
\setcounter{equation}{0}
When faced with choosing a method for 
solving a partial differential 
equation on an                      irregular domain, one 
              possibility is certainly the 
finite element method which has many attractive approximation properties
(see e.g.\ \cite{SF73}). Other choices would be
\begin{itemize}
\item use finite differences
\item give up and go home.
\end{itemize}
 As with other numerical methods for solving 
partial differential equations, using the finite element method involves
some form of \textbf{discretisation} of the problem domain. 
In the light of such observations,\\ we 
consider the general topic of % all this is ignored!!!!!
solving finite element equations with particular reference to 
producing  pretty papers.

% SECTION 2
\section{Eigenvalue bounds}
\setcounter{equation}{0}
\subsection{Analysis}
In order to analyse irregular grids we introduce lots of equations
so we can number them. Suppose that the \fe
grid contains numbered elements $e=1,\ldots,E$ with $V_e$
local unknowns on each element, giving a total of $N$
global unknowns. The coefficient matrix
$A$ can be written as
\begin{equation}
\label{assembly}
A=L^T[A_e]L\hspace{1cm}\mbox{for any matrix}\,\,A
\end{equation}
where $[A_e]$ represents an $[E.V_e \times E.V_e]$ block 
diagonal matrix whose $[V_e \times V_e]$ diagonal blocks
are the element-calculated coefficient matrices. We can now use
(\ref{assembly}) later.
\subsection{Result}
A useful result not contained in  Wathen \cite{W88} is that
 if $\{ B_e \}$ is any
set of $[V_e \times V_e]$ symmetric positive definite
matrices and a preconditioner $B$ is formed from
\begin{displaymath}
B=\left(\frac{a^2+b^2-c}{\theta \alpha}\right)
\frac{\partial x}{\partial t}\hat{\alpha}{\underline v}
\hspace{2cm}\forall\hat{\alpha}\in[0,1]
\end{displaymath}
then the eigenvalues of the global matrix $B^{-1}A$ must
satisfy
\begin{equation}
\label{ebound}
    \min_e{\lambda_{\min}(B_e^{-1}A_e)}
    \leq \lambda (B^{-1}A) \leq
    \max_e{\lambda_{\max}(B_e^{-1}A_e)}.
\end{equation}

% SECTION 3
\section{Conclusion}
\setcounter{equation}{0}
This is a load of rubbish, and the phrase 
\begin{center}
``Why bother?'' 
\end{center}
comes to mind. We list the reasons why here:\\
\begin{enumerate}
\item To enhance the education of the human race.
\item To avoid getting the sack.
\end{enumerate}

% APPENDIX
\appendix
\section{Appendix}
\setcounter{equation}{0}
In this case the stiffness matrix is 
\begin{equation}
    K_e=\frac{1}{4S_e}
    \left[ 
    \begin{array}{ccc}
    a & -b & b-a \\
    -b & c & b-c \\
    b-a & b-c & a-2b+c  
    \end{array} 
    \right]
\end{equation}
where $S_e$ is the element area, $a=y_2^2$, $b=y_1y_2$ and 
$c=x_1^2+y_1^2$.

% BIBLIOGRAPHY (number in {} is maximum number of entries allowed)
\begin{thebibliography}{99}
\bibitem{SF73}{\sc G. Strang and G.J. Fix,}
	\textit{An Analysis of the Finite Element Method,}
        Prentice-Hall, 1973.
\bibitem{W88}{\sc A.J. Wathen,}
	\textit{Spectral Bounds and Preconditioning Methods,}
         {MAFELAP VI (1987),} 
        {J.R. Whiteman, ed.,
         Academic Press, 1988, pp. 157-168.}
\end{thebibliography}

% THE END!
\end{document}
