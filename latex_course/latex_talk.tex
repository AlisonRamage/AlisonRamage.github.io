\documentclass[pdf,ramage]{prosper} 
\usepackage{color}
\usepackage{theorem}
\usepackage{fancybox}

\newcommand{\bc}{\begin{center}}
\newcommand{\ec}{\end{center}}
\newcommand{\bd}{\begin{displaymath}}
\newcommand{\ed}{\end{displaymath}}
\newcommand{\bi}{\begin{itemize}}
\newcommand{\ei}{\end{itemize}}
\newcommand{\vs}{\vspace}
\newcommand{\va}{{\bf a}}
\newcommand{\vb}{{\bf b}}
\newcommand{\ve}{{\bf e}}
\newcommand{\vf}{{\bf f}}
\newcommand{\vp}{{\bf p}}
\newcommand{\vr}{{\bf r}}
\newcommand{\vu}{{\bf u}}
\newcommand{\vv}{{\bf v}}
\newcommand{\vw}{{\bf w}}
\newcommand{\vx}{{\bf x}}
\newcommand{\vX}{{\bf X}}
\newcommand{\vy}{{\bf y}}
\newcommand{\vze}{{\bf 0}}
\newcommand{\bs}{$\backslash$}
\newcommand{\blank}{\begin{slide}{}\end{slide}}
\newcommand{\lx}{{\LaTeX} }


\slideCaption{SMSTC 2007}

\newenvironment{myslide}[1]{\begin{slide}{#1}}{\end{slide}}
% NB: no use adjusting \slidetitle: that doesn't obviate neg. \vspace.
\def\mynormal{\vspace*{-0.5cm}}
\def\mysmall{\small\vspace*{-0.5cm}}
\def\myfoot{\footnotesize\vspace*{-0.5cm}}
\def\myscript{\scriptsize\vspace*{-0.5cm}}

% This modified from seminar.cls, where it sets it to 1.2:
\def\slidearraystretch{1}

%%%%%%%%%%%%%%%%%
\def\red{\color{red}}
\def\blue{\color{blue}}
\def\green{\color{green}}
\def\orange{\color{orange}}
\def\magenta{\color{magenta}}

\definecolor{orange}{rgb}{1,0.25,0.5}
\definecolor{magenta}{rgb}{.75,0,.75}
\definecolor{green}{rgb}{0,0.65,0}
% \newcommand{\about}[1]{\centerline{\color{blue}\textbf{#1}}}
 
% % Make label in descriptions coloured.  Copied from article.cls.
% \renewcommand*\descriptionlabel[1]{\hspace\labelsep
%                                 \normalfont\bfseries\green#1}
 
%%%%%%%%%%%%%%%%%%%%%%%%%%%%%%%%%%%%%%%%%%%%%%%%%%%%%%%%%%%%%%%%%%%
\def\bordermatcolleft#1#2#3#4#5{% dims then entries, both top to bottom
\bordermatrix{   & \SS #1 \cr
      \SS\hfil   #2 & #4  \cr
         \SS     #3 & #5  \cr}}
 
\def\bordermatcolright#1#2#3#4#5{% dims then entries, both top to bottom
\begin{array}[b]{c@{}c}
\SS #1 & \\
\left[\begin{array}{@{}c@{}} #4 \\ #5 \\ \end{array}\right] &
\begin{array}{c}\SS #2 \\ \SS #3 \\\end{array}
\end{array}}

\def\bordermatcolrightthree#1#2#3#4#5#6#7{% dims then entries, both top to bottom
\begin{array}[b]{c@{}c}
\SS #1 & \\
\left[\begin{array}{@{}c@{}} #5 \\ #6 \\ #7 \\ \end{array}\right] &
\begin{array}{c}\SS #2 \\ \SS #3 \\ \SS #4 \\\end{array}
\end{array}}
%%%%%%%%%%%%%%%%%%%%%%%%%%%%%%%%%%%%%%%%%%%%%%%%%%%%%%%%%%%%%%%%%%%

\def\mytitle{
\textcolor{blue}{An Introduction to \lx  }}
\title{\mytitle}
\author{Alison Ramage}
\institution{University of Strathclyde}

\begin{document}
\begin{slide}{}
\vspace*{-1cm}
\begin{center}
\shadowbox{
\begin{minipage}{4in}
\begin{center}
\fontTitle{\mytitle}
\end{center}
\end{minipage}
}
\vspace{0.6cm}

\bigskip
\parbox{5cm}{\bc
Alison Ramage\\
Dept of Mathematics\\
University of Strathclyde\\
Glasgow, Scotland\ec}
\parbox{5cm}{\bc
\scalebox{0.8}{\includegraphics{logo_global.eps}}\\
\ec}
\vspace{0.6cm}
\end{center}
\begin{mylist1}[blue]
\item
\textcolor{green}{
{\em \lx: A Document Preparation System}}\\
Lamport, Addison-Wesley 1994 
\item
\textcolor{green}{
{\em Learning \lx}}\\
 Griffiths and  Higham, SIAM 1997
\item 
\textcolor{green}{
{\em The \lx Companion}}\\
Mittelbach and Goossens, Addison-Wesley 2004
\end{mylist1}

\end{slide}

\overlays{7}{
\begin{slide}{Basics}
\mynormal
\medskip
\begin{mylist1}[blue]
\item based on the \textcolor{green}{\TeX}\, program by Don Knuth\\
\medskip
 \fromSlide{2}{
\item \textcolor{green}{\lx} written by Leslie Lamport specifically for maths
\begin{quotation}
\textcolor{red}{
``a house built with wood and nails of \TeX''
}
\end{quotation}
}
\medskip
 \fromSlide{3}{
\item \textcolor{green}{typesetting} package (not \textcolor{red}{WYSIWYG})
 }
\medskip
\fromSlide{4}{
\item cross-references and citations done \textcolor{green}{automatically}
}
\medskip
\fromSlide{5}{
\item convenient for electronic publication
}
\medskip
\fromSlide{6}{
\item \textcolor{green}{public domain} software (i.e.\ \textcolor{red}{free}!)
}
\end{mylist1}
\fromSlide{7}{
\bc
\textcolor{red}{
``Although you can format an equation almost any way you want with \lx, you have
to work harder to do it wrong.''
}
\ec
}
\end{slide}
}
\overlays{7}{
\begin{slide}{Overview}
\mynormal
%\medskip
\bc
create a \lx .tex file\\
\textcolor{blue}{kepler\$}
\textcolor{green}{emacs doc.tex}
\ec
\fromSlide{2}{
\begin{center}
\scalebox{0.1}{\includegraphics{downarrow.eps}}
\end{center}
\bc
process the .tex file\\
\textcolor{blue}{kepler\$}
\textcolor{green}{latex doc}
\ec
}
\fromSlide{3}{
\vspace{-2cm}
\begin{flushright}
\scalebox{0.1}{\includegraphics{uparrow.eps}}
\parbox{3cm}{\textcolor{red}{correct any\\ \lx  errors}}
\end{flushright}
}
\fromSlide{4}{
\vspace{0.8cm}
\begin{center}
\scalebox{0.1}{\includegraphics{downarrow.eps}}
\end{center}
\bc
preview the .dvi file\\
\textcolor{blue}{kepler\$}
\textcolor{green}{xdvi doc}
\ec
}
\fromSlide{5}{
\vspace{-4cm}
\begin{flushleft}
\hspace*{1.2cm}\scalebox{0.1}{\includegraphics{uparrow.eps}}
\end{flushleft}
\begin{flushleft}
\parbox{3cm}{\bc\textcolor{red}{correct any\\ style, layout,\\content errors}\ec}
\end{flushleft}
\begin{flushleft}
\hspace*{1.2cm}\scalebox{0.1}{\includegraphics{uparrow.eps}}
\end{flushleft}
}
\fromSlide{6}{
\vspace{1cm}
\begin{center}
\scalebox{0.1}{\includegraphics{downarrow.eps}}
\end{center}
\bc
convert the .dvi file to PostScript\\
\textcolor{blue}{kepler\$}
\textcolor{green}{dvips doc}
\ec
}
\fromSlide{7}{
\begin{center}
\scalebox{0.1}{\includegraphics{downarrow.eps}}
\end{center}
\bc
print the .ps file\\
\textcolor{blue}{kepler\$}
\textcolor{green}{lpr doc.ps}
\ec
}
\end{slide}
}

\overlays{3}{
\begin{slide}{Observations}
\mynormal
\medskip
\begin{mylist1}[blue]
%\item A \lx file must have the extension \textcolor{green}{.tex}
\fromSlide{1}{
\item If you are using internal references, you may get the message
\begin{flushleft}
\texttt{LaTeX Warning: Label(s) may have changed.
Rerun to get cross-references right.}
\end{flushleft}
You must then \textcolor{red}{recompile} the file by using the \textcolor{green}{latex} command
again.
}
\fromSlide{2}{
\item The \textcolor{green}{previewing} stage is very important: printing is
\textcolor{red}{expensive}!
}
\fromSlide{3}{
\item The \textcolor{green}{dvips} command can be used to print selected pages,
e.g.
\bc
\textcolor{blue}{kepler\$}
\textcolor{green}{dvips -p29 -l34 thesis}
\ec
prints pages 29-34 of the document thesis.dvi.
}
\end{mylist1}
\end{slide}
}

\overlays{4}{
\begin{slide}{In your .tex file}
\mynormal
\medskip
\begin{mylist1}[blue]
\item\textcolor{red}{ALL} \lx commands begin with a \,\,
$\textcolor{green}{\backslash}$\,\,
and are case-sensitive.
\fromSlide{2}{
\item Special characters:
\bc
\textcolor{green}{
\% $\ast$ \$ \& \_ \{ \} \~{} \^{} $\backslash$}
\ec
will be interpreted as \lx control characters.
}
\fromSlide{3}{
\item \textcolor{green}{\%} acts as a `comment' symbol: anything on the line
after a \textcolor{green}{\%} sign is ignored.
}
\fromSlide{4}{
\item Some commands have \textcolor{green}{arguments}:

\begin{center}
enclosed in \textcolor{red}{$\{\,\,\}$}: \textcolor{green}{compulsory}\\
\end{center}
\begin{center}
enclosed in \textcolor{red}{$[\,\,]$}: \textcolor{green}{optional}\\
\end{center}
}
\end{mylist1}
\end{slide}
}

\begin{slide}{Document structure}
\mynormal
\medskip
Each document has two parts:

\begin{mylist1}[blue]
\item \textcolor{red}{PREAMBLE}\\
This sets up the document style, type size, 
page settings etc. The first line must be
\begin{verbatim}
\documentclass{STYLE}
\end{verbatim}
where STYLE is one of
\begin{mylist1}[red]
\item \textcolor{green}{article} (includes sections and subsections)
\item                \textcolor{green}{report}  (includes chapters)
    \item            \textcolor{green}{book}    (includes volumes)
        \item        \textcolor{green}{letter}
            \item    \textcolor{green}{slides}
\item \textcolor{green}{\ldots}
\end{mylist1}
\end{mylist1}
\end{slide}

\overlays{4}{
\begin{slide}{Preamble (cont.)}

\mynormal
\medskip
The {preamble} may also contain 
\begin{mylist1}[red]
\item commands which define page size, margins etc,
e.g.

{\tt
$\backslash$setlength\{$\backslash$textheight\}\{18.0cm\}\\
$\backslash$setlength\{$\backslash$topmargin\}\{1.2in\}\\
$\backslash$pagestyle\{empty\}\\ }
\fromSlide{2}{
\item inclusion of any \textcolor{green}{packages}, e.g.

{\tt 
$\backslash$usepackage\{amssymb\}\\ }
}
\fromSlide{3}{
\item user-defined \textcolor{green}{new commands}, e.g.

{\tt 
$\backslash$newcommand\{$\backslash$fe\}\{finite element method\}\\ }
}
\fromSlide{4}{
\item user-defined \textcolor{green}{changes} to default style, e.g.

{\tt 
$\backslash$renewcommand\{$\backslash$baselinestretch\}\{1.5\}\\ }
}
\end{mylist1}
\end{slide}
}

\overlays{2}{
\begin{slide}{Document structure (cont.)}
\mynormal
\medskip
\begin{mylist1}[blue]
\item \textcolor{red}{DOCUMENT BODY}\\
This contains the \textcolor{green}{\lx} commands to produce the document text.\\
\medskip
{\tt
$\backslash$begin\{document\}             \\
THE DOCUMENT TEXT GOES HERE.\\
$\backslash$end\{document\}\\
}
\medskip
\fromSlide{2}{
\begin{mylist1}[red]
\item 
A document can be \textcolor{green}{subdivided} using\\
\medskip

{\tt
$\backslash$chapter\{...\}\\
$\backslash$section\{...\}\\
$\backslash$subsection\{...\}\\
$\backslash$subsubsection\{...\}\\
$\backslash$Appendix
}
\end{mylist1}
}
\end{mylist1}
\end{slide}
}

\begin{slide}{Document structure (cont.)}
\mynormal
\medskip
\begin{mylist1}[red]
\item 
Separate \LaTeX \,\,input files can be \textcolor{green}{included} using\\
\medskip
{\tt
$\backslash$input\{\textit{filename}\}}\\
\medskip
\item 
A \textcolor{green}{table of contents} can be included using\\
\medskip
{\tt
$\backslash$tableofcontents}\\
\bigskip
\bigskip
Sample document body:\\
\bigskip
{\tt
$\backslash$tableofcontents\\
$\backslash$newpage\\
 $\backslash$input\{chapter1\}\\
$\backslash$newpage\\
 $\backslash$input\{chapter2\}\\
}
\end{mylist1}
\end{slide}


\overlays{3}{
\begin{slide}{SUMMARY}
\mynormal
\medskip
\begin{mylist1}[blue]
\item A valid \lx  document:\\
\medskip
{\tt
$\backslash$documentclass\{article\}\\
$\backslash$begin\{document\}\\
This is my first document.\\
$\backslash$end\{document\}\\}
\end{mylist1}
\bigskip
\fromSlide{2}{
\bc
\textcolor{green}{This is all you need, but\ldots}
\ec
}
\fromSlide{3}{
\bc
\textcolor{green}{\ldots hopefully your thesis will be a little more
sophisticated!}
\ec
}
\end{slide}
}
\overlays{3}{
\begin{slide}{Document style}
\mynormal
\begin{mylist1}[red]
\item A \textcolor{green}{title} can be created with
{\tt $\backslash$maketitle} which uses
\bc
{\tt\bs title\{...\}\qquad
\bs author\{...\}\qquad
\bs date\{\bs today\} }
\ec
\medskip
\fromSlide{2}{
\item Varying text \textcolor{green}{font}:
\bc
\textit{\bs textit\{...\}}\qquad
\textsc{\bs textsc\{...\}}\qquad 
\textsf{\bs textsf\{...\}}\\
\textbf{\bs textbf\{...\}}\qquad 
\texttt{\bs texttt\{...\}}
\ec
}
\medskip
\fromSlide{3}{
\item Varying text \textcolor{green}{size}:
\bc{\tt
\tiny \bs tiny\qquad \scriptsize\bs scriptsize\qquad \footnotesize\bs footnotesize\qquad 
\normalsize \bs normalsize \qquad
\large \bs large\qquad 
\Large \bs Large\qquad \LARGE \bs LARGE\qquad \huge \bs huge\qquad \Huge\bs Huge}
\ec
}
\end{mylist1}
\end{slide}
}

\begin{slide}{Some useful \lx concepts (1)}
\mynormal
\medskip
\begin{mylist1}[blue]
\item \textcolor{green}{environments}
\begin{verbatim}
\begin{center}...\end{center}
\begin{itemize}...\end{itemize}
\begin{enumerate}...\end{enumerate}
\begin{description}...\end{description}
\begin{tabbing}...\end{tabbing}
\begin{tabular}...\end{tabular}
\begin{table}...\end{table}
\begin{figure}...\end{figure}
\begin{quote}...\end{quote}
\begin{verse}...\end{verse}
\end{verbatim}
\end{mylist1}
\end{slide}

\overlays{2}{
\begin{slide}{Some useful \lx concepts (2)}
\mynormal
\medskip
\begin{mylist1}[blue]
\item \textcolor{green}{math mode}\\

All mathematics must be typeset in \textcolor{green}{math mode}. This can be done 
using \textcolor{red}{dollar signs} \textcolor{green}{\$...\$}, e.g.
\bc
{\tt Let \$x\$ be a real number.}\\
\medskip
 Let $x$ be a real number.
\ec
\medskip
\fromSlide{2}{
\item \textcolor{green}{environments in math mode}\\
\bc
\tt
$\backslash$begin\{displaymath\}...$\backslash$end\{displaymath\}\\
$\backslash$begin\{equation\}...$\backslash$end\{equation\}\\
$\backslash$begin\{array\}...$\backslash$end\{array\}\\
$\backslash$begin\{eqnarray\}...$\backslash$end\{eqnarray\}\\
$\backslash$begin\{eqnarray$\ast$\}...$\backslash$end\{eqnarray$\ast$\}\\
\ec
}
\end{mylist1}
\end{slide}
}


\overlays{4}{
\begin{slide}{Examples of maths commands (1)}
\mynormal
\begin{mylist1}[blue]
\item \textcolor{green}{maths fonts}:
\bc
$\mathcal{A}$\qquad
$\mathrm{A}$\qquad
$\mathit{A}$\qquad
$\mathsf{A}$\qquad
$\mathbf{A}$\qquad
$\mathtt{A}$\\
{\tt
{\bs mathcal\{A\}},
{\bs mathrm\{A\}},
{\bs mathit\{A\}},\\
{\bs mathsf\{A\}},
{\bs mathbf\{A\}},
{\bs mathtt\{A\}}}
\ec
\fromSlide{2}{
\medskip
\item \textcolor{green}{Greek letters}:\qquad
$\alpha,\quad\beta, \quad \gamma, \quad\Gamma, \quad\sigma, \quad\Sigma$
\bc
{\tt
\bs alpha,\bs beta,\bs gamma,\bs Gamma,\bs sigma,\bs Sigma
}
\ec
}
\fromSlide{3}{
\medskip
\item \textcolor{green}{symbols}:\qquad
$\ne,\quad\Leftrightarrow,\quad\in ,\quad\sim,\quad\nabla$
\bc
{\tt
\bs ne,\bs Leftrightarrow,\bs in,\bs sim,\bs nabla
}
\ec
}
\fromSlide{4}{
\medskip
\item \textcolor{green}{variable-sized symbols}:\qquad
$\int,\quad\oint,\quad\sum,\quad\left[,\quad\right]$
\bc
{\tt 
\bs int,\quad\bs oint,\quad\bs sum,\quad\bs left[, \quad\bs right]}
\ec
}
\end{mylist1}
\end{slide}
}
\overlays{2}{
\begin{slide}{Examples of maths commands (2)}
\mynormal
\medskip
\begin{mylist1}[blue]
\item \textcolor{green}{subscripts} and \textcolor{green}{superscripts}:\qquad
\bd
x_1,\quad y_{ij},\quad z^{n+1},\quad\lim_{x\rightarrow -1},\quad \int_1^\infty
\ed
\bc
{\tt x\_1,\quad y\_\{ij\},\quad z\^{}\{n+1\},\\
\quad\bs lim\_\{x\bs rightarrow -1\},\quad \bs int\_1\^{}\bs infty }
\ec
\bigskip
\fromSlide{2}{
\item \textcolor{green}{fractions}:
\bd
x=\frac{3+\sin{t}}{t^2},\qquad y=\frac{\partial x}{\partial t}
\ed
\bc
{\tt
x=\bs frac\{3+\bs sin\{t\}\}\{t\^{}2\}\\ y=\bs frac\{\bs partial x\}\{\bs partial t\}}
\ec
}
\end{mylist1}
\end{slide}
}
\overlays{2}{
\begin{slide}{Arranging formulae}
\mynormal
\medskip
\begin{mylist1}[blue]
\item \textcolor{green}{arrays}:\qquad
$A=\left[\begin{array}{ccc}
1 & 1 & 1\\
x & y & z\\
x^2 & y^2 & z^2
\end{array}\right]$\\
\bigskip
{\tt
A=\bs left[\bs begin\{array\}\{ccc\}\\
1 \& 1 \& 1\bs \bs  x \& y \& z\bs \bs x\^{}2 \& y\^{}2 \& z\^{}2\\
 \bs end\{array\} \bs right]
}
\fromSlide{2}{
\medskip
\item \textcolor{green}{equation arrays}:
\vspace{-0.5cm}
\begin{eqnarray}
x&=&17+p^2-3p^5\\
y&=&\alpha -\theta\nonumber
\end{eqnarray}
{\tt \bs begin\{eqnarray\}\\
x\&=\&17+p\^{}2-3p\^{}5\bs \bs \\
y\&=\&\bs alpha - \bs theta\bs nonumber\\
\bs end\{eqnarray\}}
}
\end{mylist1}
\end{slide}
}
\begin{slide}{A simple table}
\mynormal
\medskip
\begin{center}
\begin{tabular}{||l||c|c|c|c|c|c|}
\hline
Team & Played & W & D & L & Goals & Points\\\hline\hline
Aberdeen & 2 & 2 & 0 & 0 & +10 & 6\\\hline
Celtic & 2 & 0 & 1 & 1& -5 & 1 \\\hline
Rangers & 2 & 0 & 1 & 1& -5 & 1 \\\hline
\end{tabular}
\end{center}
\medskip
\begin{verbatim}
\begin{center}
\begin{tabular}{||l||c|c|c|c|c|c|}
\hline
Team & Played & W & D & L & Goals & Points\\
\hline\hline
Aberdeen & 2 & 2 & 0 & 0 & +10 & 6\\\hline
Celtic & 2 & 0 & 1 & 1& -5 & 1 \\\hline
Rangers & 2 & 0 & 1 & 1& -5 & 1 \\\hline
\end{tabular}
\end{center}
\end{verbatim}
\end{slide}

\overlays{2}{
\begin{slide}{Cross-referencing}
\mynormal
\medskip
\lx can \textcolor{red}{automatically} number equations, references etc
\begin{mylist1}[blue]
\medskip
\item to label an equation: \qquad\textcolor{green}{\tt \bs label\{...\}}\\
\begin{equation}
x=1
\label{eq1}
\end{equation}
{\tt
\bs begin\{equation\}\\
\textcolor{green}{\bs label\{eq1\}}\\
x=1\\
\bs end\{equation\}}
\fromSlide{2}{
\medskip
\item to refer to a label:\qquad\textcolor{green}{\tt \bs ref\{...\}}\\
\medskip
\bc
Using equation (\ref{eq1}), we see that\ldots
\ec
{\tt
Using equation (\textcolor{green}{\bs ref\{eq1\}}), we see that\ldots
}
}
\end{mylist1}
\end{slide}
}
\overlays{4}{
\begin{slide}{Citing reference texts}
\mynormal
\medskip
\begin{mylist1}[blue]
\item  bibliography:\\
\medskip
{\tt
\textcolor{green}{
\bs begin\{thebibliography\}\{99\}}\\
BIBLIOGRAPHY ITEMS\\
\textcolor{green}{
\bs end\{thebibliography\}}
}
\fromSlide{2}{
\medskip
\item sample bibliography entry:\\
\medskip{\tt
\textcolor{green}{
\bs bibitem\{Ramage04\} \{\\
\textcolor{red}{A. Ramage, Famous Book, OUP, 2004.}\\
\textcolor{green}{\}}\\
} }}
\medskip
\fromSlide{3}{
\item to refer to a reference text:\qquad\textcolor{green}{\tt \bs cite\{...\}}\\
\medskip
{\tt In \textcolor{green}{\bs cite\{Ramage04\}} we see that \ldots} }
\fromSlide{4}{
\medskip
\item fancier methods available, e.g. \textcolor{red}{bibtex}
}
\end{mylist1}
\end{slide}
}
\overlays{3}{
\begin{slide}{Including pictures}
\mynormal
\begin{mylist1}[blue]
\item most commonly,  plots are in \textcolor{green}{PostScript} (.ps) or
\textcolor{green}{Encapsulated PostScript} (.eps) files
\fromSlide{2}{
\item many packages available, e.g. 
\textcolor{green}{psfig}, \textcolor{green}{epsf} etc
}
\fromSlide{3}{
\item \textsc{Example}
\begin{mylist1}[red]
\item in preamble:\\
\texttt{\bs usepackage[dvips]\{graphics\}}
\item in text:\\
{\tt \bs begin\{figure\}[ht]\\
\bs begin\{center\}\\
\bs scalebox\{0.3\}\{\bs includegraphics\{fig.eps\}\}\\ 
\bs end\{center\}\\
\bs caption\{An example of including a picture.   \bs label\{fig1\}\}\\
\bs end\{figure\}}
\end{mylist1}
}
\end{mylist1}
\end{slide}
}
\overlays{2}{
\begin{slide}{Slides and presentations}
\mynormal
\medskip
\begin{mylist1}[blue]
\item \textcolor{green}{slides} document style
\bc
{\tt \bs documentclass\{slides\} }
\ec
\bc
{\tt \bs begin\{slide\}...\bs end\{slide\}}
\ec
\fromSlide{2}{
\bigskip
\item many more sophisticated packages available, e.g.
\medskip
\bc
\textcolor{red}{prosper}
\ec
\bc
{\tt http://prosper.sourceforge.net}
\ec
\bc
{\tt http://www.ma.man.ac.uk/$\sim$mheil/Prosper/}
\ec
}
\end{mylist1}
\end{slide}
}
\overlays{1}{
\begin{slide}{Packages and style files}
\mynormal
\medskip
\begin{mylist1}[blue]
\item \textcolor{green}{.sty}, \textcolor{green}{.cls} files available from many
sources:
\bigskip
\begin{mylist1}[red]
\item colleagues and fellow students
\medskip
\item publishers, e.g. 
\textcolor{green}{siamltex},
\textcolor{green}{elsart}
\medskip
\item American Mathematical Society, e.g.
\textcolor{green}{amsfonts},
\textcolor{green}{amsmath},
\textcolor{green}{amssymb}
\bc
\textcolor{red}{
$\mathbb{R}$, $\mathbb{Z}$, $\mathbb{C}$\\
{\tt \$\bs mathbb\{R\}\$,
\$\bs mathbb\{Z\}\$,
\$\bs mathbb\{C\}\$}}
\ec
\medskip
\item UK\TeX\,\,  archive\qquad {\tt http://www.tex.ac.uk}
\medskip
\item Google search!
\end{mylist1}
\end{mylist1}
\end{slide}
}
\overlays{1}{
\begin{slide}{Support Material}

Available from:\\

\bigskip

\begin{tabbing}
xxxxxxxxxxxxxxxxxxxxxxx\= \kill
\texttt{http://www.maths.strath.ac.uk}\\
\>\texttt{/$\sim$caas63/latex\_course}
\end{tabbing}

\bigskip
\bigskip
\bigskip

\begin{tabbing}
xxxxxxxxxxxxxxxxxxxxxxxxxxxx\= \kill
\textcolor{green}{$\bullet$} \lx notes \>  
\texttt{wpnotes.tex}\\
\textcolor{green}{$\bullet$} sample file \>  
\texttt{wpdoc.tex}\\
\textcolor{green}{$\bullet$} sample figure \>  
\texttt{fig1.eps}\\
\textcolor{green}{$\bullet$} slides from this talk \>  
\texttt{latex\_talk.tex}\\
\textcolor{green}{$\bullet$} Prosper document class \> 
\texttt{prosper.cls}\\
\textcolor{green}{$\bullet$} sample Prosper style file \> 
\texttt{PPRramage.sty}\\
\end{tabbing}

\end{slide}
}
\end{document}
